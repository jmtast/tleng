\begin{itemize}
\item Para mejorar la descripci\'on y entendimiento de algunas expresiones regulares que eran innecesariamente extensas, se decidi\'o modificar algunas expresiones regulares correspondientes a tokens, como así tambien se modificó la gram\'atica y su implementaci\'on. Anteriormente, había ciertos casos en que no se capturaban correctamente ciertas reglas. Por ejemplo la regla \textbf{ballon}, ya que el prefijo \textbf{ball} coincide con una primitiva, y era considerado como tal. \\
Ahora, la modificaci\'on realizada permite tomar un string completo $([a-zA-Z]+|\$|\_)$, como por ejemplo \texttt{ballon} y ver si se corresponde con una primitiva. En este caso no se corresponde por lo que puede ser nombre de una regla (y es considerado como un token \textbf{RULE\_NAME}).

\item Respecto al c\'odigo se realizaron arreglos para que funcione con los ejemplos de programa v\'alidos que se muestran en la presentaci\'on de este trabajo pr\'actico. Por ejemplo hab\'ia un problema con las definiciones de las rotaciones, que ahora se puede ver que funciona como se deseaba acorde a la presentaci\'on.
\\
Se modific\'o el c\'odigo que realiza la visualizaci\'on, ya que anteriormente ten\'ia problemas con la m\'axima profundidad.
\\
Tambi\'en se arregl\'o la forma de multiplicar las matrices en los \texttt{getters} de las clases creadas para la visualizaci\'on. La descripci\'on de las mismas se encuentra en la secci\'on de implementaci\'on.


\item Cambios en la gram\'atica: Se elimin\'o el no terminal \textbf{INITIAL\_AUXILIAR} y las producciones que lo utilizaban, se modificaron las expresiones regulares para algunos tokens y algunas producciones.

\end{itemize}


%~ Para emprolijar algunas expresiones regulares que qnos habian quedado innecesariamente extensas decidimos, modificar la gramatica asociando en keywords algunos tokens. Por ejemplo Ball Box  y void pasaron a ser keywords_tipo y las transformaciones pasaron a estar asociadas en un keywords_transformation.
%~ De esta manera la lista de tokens queda mas reducible y facil de leer, y lo mismo para sus respectivas expresiones regulares.
%~ En particular cambio mucho la definicion de regla, pues ahora toma todo un string de caracteres y luego verifica que sea un keyword.
%~ 
%~ Respecto al codigo hubo varios arreglos, pues por ejemplo las rotaciones estaban mal, luego, se refactorizo el c\'odigo del show, para un codigo mas reducible y facil de leer.
%~ la forma de multiplicar de las matrices en la clase de transformation en los geters 
%~ se arreglo el calculo de la profundidad.
%~ 
%~ Se saco el simbolo distinguido "initial_auxiliar" y las respectivas producciones que lo usaban, y esto se ve reflejado en el archivo yacc.py
%~ 
%~ Ahora el simbolo distinguido es initial
%~ 
%~ Cambiamos el nombre del token por uname para que se entienda que solo es un nombre
