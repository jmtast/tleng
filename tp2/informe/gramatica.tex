\section{Gramática}

La Gramática desarrolada es la siguiente: $\langle V_n,\ V_t,\ Producciones,\ Start \rangle$

\subsection{Símbolos No Terminales}

Start, TempoDefinition, BarDefinition, Constants, Voices, MaybeVoices, Voice, Bars, MaybeBars, Bar, Repeat, Notes, MaybeDot, MaybeNoteModifier.

\subsection{Símbolos Terminales}

\#tempo, \#compas, '/', const, const\_name, figure, number, ';', voz, '\(', '\)', '\{', '\}', compas, repetir, silencio, nota, note, '.', ',', note\_modifier.

\subsection{Producciones}
Estas son las producciones utilizadas por nuestro parser para generar el \textit{AST} de la cadena de entrada. En \textbf{negrita} se resaltan los símbolos \textit{no terminales} para facilitar la lectura. \\
\\
Se decidió requerir que exista al menos una voz y un compás por voz a nivel de la Gramática. \\
\\
\begin{tabular}{l l l}
	\textbf{Start} & $\rightarrow$ & \textbf{TempoDefinition} \textbf{BarDefinition} \textbf{Constants} \textbf{Voices} \\
	\textbf{TempoDefinition} & $\rightarrow$ & \#tempo figure number \\
	\textbf{BarDefinition} & $\rightarrow$ & \#compas number/number \\
	\textbf{Constants} & $\rightarrow$ & $\lambda$ $\vert$ const const\_name=number; \textbf{Constants} \\
	\textbf{Voices} & $\rightarrow$ & \textbf{Voice} \textbf{MaybeVoices} \\
	\textbf{MaybeVoices} & $\rightarrow$ & $\lambda$ $\vert$ \textbf{Voice} \textbf{MaybeVoices} \\
	\textbf{Voice} & $\rightarrow$ & voz (number) { bars } $\vert$ voz (const\_name) { \textbf{Bars} } \\
	\textbf{Bars} & $\rightarrow$ & \textbf{Bar} \textbf{MaybeBars} \\
	\textbf{MaybeBars} & $\rightarrow$ & $\lambda$ $\vert$ \textbf{Bar} \textbf{MaybeBars} \\
	\textbf{Bar} & $\rightarrow$ & \textbf{Repeat} $\vert$ compas { \textbf{Notes} } \\
	\textbf{Repeat} & $\rightarrow$ & repetir (number) { bars } $\vert$ repetir (const\_name) { bars } \\
	\textbf{Notes} & $\rightarrow$ & $\lambda$ $\vert$ silencio (figure \textbf{MaybeDot}); \textbf{Notes} $\vert$ \\
	& & note\_call (note \textbf{MaybeNoteModifier}, number, figure \textbf{MaybeDot}); \textbf{Notes} $\vert$ \\
	& & note\_call (note \textbf{MaybeNoteModifier}, const\_name, figure \textbf{MaybeDot}); \textbf{Notes} \\
	\textbf{MaybeDot} & $\rightarrow$ & $\lambda$ $\vert$ . \\
	\textbf{MaybeNoteModifier} & $\rightarrow$ & $\lambda$ $\vert$ note\_modifier \\
\end{tabular}

\subsection{Expresiones regulares de los tokens}

El lado izquierdo de cada una de las siguientes líneas es el valor de un \textit{token} dado, y el lado derecho es la expresión regular que reconoce las cadenas que luego se transforman en el mismo al hacer la primera pasada del código con el lexer. \\

\begin{tabular}{l l l}
	note\_modifier & = & + $\vert$ - \\
	const\_name & = & [a-zA-Z][a-zA-Z0-9]* \\
	number & = & [0-9]+ \\
	figure & = & redonda $\vert$ blanca $\vert$ negra $\vert$ corchea $\vert$ semicorchea $\vert$ fusa $\vert$ semifusa \\
	note & = & do $\vert$ re $\vert$ mi $\vert$ fa $\vert$ sol $\vert$ la $\vert$ si \\
\end{tabular}

\clearpage
