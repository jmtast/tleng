\section{Solución del Problema}

Para crear el traductor se dividió la solución en cuatro partes: una Gramática, un Lexer, un Parser y un Traductor.

Se programó con \textbf{Python} utilizando la librería \textbf{Ply}.

\subsection{Lexer}

El Lexer traduce la cadena original a una cadena tokenizada asignando valores a cada token.

\begin{itemize}
	\item Los siguientes tokens se traducen sin cambios: '(', ')', '\{', '\}', ';', ',', '=', '/', '.', 'voz', 'repetir', 'compas', '\#tempo', '\#compas', 'const', 'nota', 'silencio'.

	\item Los números se traducen al token $NUMBER$.

	\item Las figuras: 'redonda', 'blanca', 'negra', 'corchea', 'semicorchea', 'fusa' y 'semifusa' se traducen al token $FIGURE$.

	\item Las notas: 'do', 're', 'mi', 'fa', 'sol', 'la' y 'si' se traducen al token $NOTE$.

	\item Los símbolos '+' y '-' se traducen al token $NOTE\_MODIFIER$

	\item Los comentarios iniciando con '//' hasta el fin de línea se eliminan.

	\item Y cualquier otra cadena que comience con una letra y contenga únicamente letras, números y '\_' se traduce a $CONST\_NAME$
\end{itemize}

A continuación se muestra el código de $lexer\_rules.py$ que define estos tokens para \textbf{Ply}:

\begin{verbatim}
# -*- coding: utf-8 -*-
tokens = [
    'TEMPO_DEFINITION',
    'BAR_DEFINITION',
    'CONST_DEFINITION',
    'FIGURE',
    'NUMBER',
    'SLASH',
    'EQUAL',
    'SEMICOLON',
    'LPAREN',
    'RPAREN',
    'LBRACE',
    'RBRACE',
    'VOICE_BLOCK',
    'REPEAT_BLOCK',
    'BAR_BLOCK',
    'NOTE_CALL',
    'SILENCE',
    'DOT',
    'NOTE',
    'NOTE_MODIFIER',
    'CONST_NAME',
    'COMMA'
]

t_LPAREN = r"\("
t_RPAREN = r"\)"
t_LBRACE = r"\{"
t_RBRACE = r"\}"
t_SEMICOLON = r"\;"
t_EQUAL = r"\="
t_SLASH = r"\/"
t_VOICE_BLOCK = r"voz"
t_REPEAT_BLOCK = r"repetir"
t_BAR_BLOCK = r"compas"
t_TEMPO_DEFINITION = r"\#tempo"
t_BAR_DEFINITION = r"\#compas"
t_CONST_DEFINITION = r"const"
t_NOTE_CALL = r"nota"
t_SILENCE = r"silencio"
t_DOT = r"\."
t_NOTE_MODIFIER = r"(\+|\-)"
# Cualquier nombre excepto las keywords de la gramática
t_CONST_NAME = r"(?!(const|voz|nota|repetir|compas|silencio))[a-zA-Z][_a-zA-Z0-9]*"
t_COMMA = r","

t_ignore = " \t"

def t_NUMBER(token):
    r"[0-9]+"
    token.value = int(token.value)
    return token

def t_FIGURE(token):
    r"(redonda|blanca|negra|corchea|semicorchea|fusa|semifusa)"
    return token

def t_NOTE(token):
    r"(?!(redonda|repetir|silencio))(do|re|mi|fa|sol|la|si)"
    # La negación es necesaria porque sino toma las notas primero
    # Ya se probó cambiar el orden y no funcionó
    return token

def t_NEWLINE(token):
    r"\n+"
    token.lexer.lineno += len(token.value)

def t_IGNORE_COMMENTS(token):
    r"//(.*)\n"
    token.lexer.lineno += 1

def t_error(token):
    raise Exception("Error de sintaxis: Token desconocido en línea {0}. \"{1}\"".format(
        token.lineno, token.value.partition("\n")[0]))

\end{verbatim}

\newpage

\subsection{Parser}

El Parser toma la cadena ya tokenizada y genera el \textbf{Abstract Syntax Tree}.

Las producciones especificadas son las de la Gramática, cada función equivale a una producción (sin utilizar pipes '$\vert$').

A continuación se muestra el código de $parser\_rules.py$ que define estas producciones para \textbf{Ply}:

\begin{verbatim}
# -*- coding: utf-8 -*-
from lexer_rules import tokens

from expressions import *

constants = {}

def p_start(subexpressions):
    'start : tempo_definition bar_definition constants voices'
    subexpressions[0] = Start(subexpressions[1], subexpressions[2], subexpressions[4])

def p_tempo_definition(subexpressions):
    'tempo_definition : TEMPO_DEFINITION FIGURE NUMBER'
    subexpressions[0] = TempoDefinition(subexpressions[2], subexpressions[3], subexpressions.lineno(1))

def p_bar_definition(subexpressions):
    'bar_definition : BAR_DEFINITION NUMBER SLASH NUMBER'
    subexpressions[0] = BarDefinition(subexpressions[2], subexpressions[4], subexpressions.lineno(1))

def p_constants_empty(subexpressions):
    'constants :'
    pass

def p_constants(subexpressions):
    'constants : CONST_DEFINITION CONST_NAME EQUAL NUMBER SEMICOLON constants'
    name = subexpressions[2]
    value = subexpressions[4]

    if name in constants:
        raise Exception("Constante redefinida: {0}. Primera vez definida en línea {1}".format(
            name, subexpressions.lineno(1)))
    constants[name] = value

# Las siguientes producciones:
#     Voices -> Voice MaybeVoices
#     MaybeVoices -> lambda | Voice MaybeVoices
# Son para requerir que exista 1 voz al nivel de la gramática

def p_voices(subexpressions):
    'voices : voice maybe_voices'
    subexpressions[0] = Voices(subexpressions[1], subexpressions[2])

def p_maybe_voices_empty(subexpressions):
    'maybe_voices :'
    pass

def p_maybe_voices(subexpressions):
    'maybe_voices : voice maybe_voices'
    subexpressions[0] = Voices(subexpressions[1], subexpressions[2])

def p_voice_number(subexpressions):
    'voice : VOICE_BLOCK LPAREN NUMBER RPAREN LBRACE bars RBRACE'
    subexpressions[0] = Voice(subexpressions[3], subexpressions[6], subexpressions.lineno(1))

def p_voice_constant(subexpressions):
    'voice : VOICE_BLOCK LPAREN CONST_NAME RPAREN LBRACE bars RBRACE'
    if subexpressions[3] not in constants:
        raise Exception("Constante no definida: {0}. Utilizada en línea {1}".format(
            subexpressions[3], subexpressions.lineno(1)))
    subexpressions[0] = Voice(constants[subexpressions[3]], subexpressions[6], subexpressions.lineno(1))

# Con Bars se hace lo mismo que con Voices para requerir al menos un compas

def p_bars(subexpressions):
    'bars : bar maybe_bars'
    subexpressions[0] = Bars(subexpressions[1], subexpressions[2])

def p_maybe_bars_empty(subexpressions):
    'maybe_bars :'
    pass

def p_maybe_bars(subexpressions):
    'maybe_bars : bar maybe_bars'
    subexpressions[0] = Bars(subexpressions[1], subexpressions[2])

def p_bar_repeat(subexpressions):
    'bar : repeat'
    subexpressions[0] = subexpressions[1]

def p_bar(subexpressions):
    'bar : BAR_BLOCK LBRACE notes RBRACE'
    subexpressions[0] = Bar(subexpressions[3], subexpressions.lineno(1))

def p_repeat_number(subexpressions):
    'repeat : REPEAT_BLOCK LPAREN NUMBER RPAREN LBRACE bars RBRACE'
    subexpressions[0] = Repeat(subexpressions[3], subexpressions[6], subexpressions.lineno(1))

def p_repeat_constant(subexpressions):
    'repeat : REPEAT_BLOCK LPAREN CONST_NAME RPAREN LBRACE bars RBRACE'
    if subexpressions[3] not in constants:
        raise Exception("Constante no definida: {0}. Utilizada en línea {1}".format(
            subexpressions[3], subexpressions.lineno(1)))
    subexpressions[0] = Repeat(constants[subexpressions[3]], subexpressions[6], subexpressions.lineno(1))

def p_notes_empty(subexpressions):
    'notes :'
    pass

def p_notes_silence(subexpressions):
    'notes : SILENCE LPAREN FIGURE maybe_dot RPAREN SEMICOLON notes'
    subexpressions[0] = Silence(subexpressions[3], subexpressions[4], subexpressions[7])

def p_notes_note_number(subexpressions):
    'notes : NOTE_CALL LPAREN NOTE maybe_note_modifier COMMA NUMBER COMMA FIGURE maybe_dot RPAREN
        SEMICOLON notes'
    subexpressions[0] = Note(subexpressions[3], subexpressions[4], subexpressions[6], subexpressions[8],
        subexpressions[9], subexpressions[12], subexpressions.lineno(1))

def p_notes_note_constant(subexpressions):
    'notes : NOTE_CALL LPAREN NOTE maybe_note_modifier COMMA CONST_NAME COMMA FIGURE maybe_dot RPAREN
        SEMICOLON notes'
    if subexpressions[6] not in constants:
        raise Exception("Constante no definida: {0}. Utilizada en línea {1}".format(
            subexpressions[6], subexpressions.lineno(1)))
    subexpressions[0] = Note(subexpressions[3], subexpressions[4], constants[subexpressions[6]],
        subexpressions[8], subexpressions[9], subexpressions[12], subexpressions.lineno(1))

def p_maybe_dot_empty(subexpressions):
    'maybe_dot :'
    pass

def p_maybe_dot(subexpressions):
    'maybe_dot : DOT'
    subexpressions[0] = Dot(subexpressions[1])

def p_maybe_note_modifier_empty(subexpressions):
    'maybe_note_modifier :'
    pass

def p_maybe_note_modifier(subexpressions):
    'maybe_note_modifier : NOTE_MODIFIER'
    subexpressions[0] = NoteModifier(subexpressions[1])

def p_error(subexpressions):
    raise Exception("Error de sintaxis en línea {0}".format(subexpressions.lineno))

\end{verbatim}

\newpage

\subsection{Traductor}

El traductor recibe el \textbf{Abstract Syntax Tree} que, en este caso, equivale a una instancia de la clase \textbf{Start} y lo recorre para ir traduciendo al lenguaje intermedio para que \textbf{midicomp} lo pueda traducir a $MIDI$.

Se decidió que dónde se espera un bloque compás (clase $Bar$) pueda ir un bloque de repetir (clase $Repeat$), permitiendo repetir anidadamente con el resultado esperado.

La clase $Repeat$ entonces responde al método $get_arr_bars()$ para devolver los compases hijos, repetiéndolos las veces correspondientes.

También se decidió que dónde se espera una nota (clase $Note$), pueda ir un silencio (clase $Silence$).

Luego se chequeará si es una nota para escribir en el track, pero ambos devuelven su valor para aumentar el tiempo transcurrido.

A continuación se muestra el código de $expressions.py$ que contiene las clases que conforman el \textbf{AST}:

\begin{verbatim}
# -*- coding: utf-8 -*-
figure_values = {
    "redonda": 1,
    "blanca": 2,
    "negra": 4,
    "corchea": 8,
    "semicorchea": 16,
    "fusa": 32,
    "semifusa": 64
}

class Start(object):
    def __init__(self, tempo, bar, voices):
        self.tempo = tempo
        self.bar = bar
        self.voices = voices

        arr_voices = self.get_voices()
        if len(arr_voices) > 16:
            raise Exception(
                "No se permiten más de 16 voces. Línea {0}".format(arr_voices[16].line))

        for voice in arr_voices:
            for bar in voice.get_bars():
                if bar.get_value() < self.bar.get_value():
                    raise Exception(
                        "Compás con duración ({0}) más corta que la indicada ({1}). Línea {2}".format(
                            bar.get_value(), self.bar.get_value(), bar.line))
                if bar.get_value() > self.bar.get_value():
                    raise Exception(
                        "Compás con duración ({0}) más larga que la indicada ({1}). Línea {2}".format(
                            bar.get_value(), self.bar.get_value(), bar.line))

    def name(self):
        return "start"

    def children(self):
        return [self.tempo, self.bar, self.voices]

    def get_voices(self):
        return self.voices.get_arr_voices();

class TempoDefinition(object):
    def __init__(self, figure, speed, line):
        self.figure = figure
        self.speed = speed

        if speed == 0:
            raise Exception(
                "Cantidad de repeticiones por minuto incorrecta. Debe ser mayor a 0. Línea {0}".format(
                    line))

    def name(self):
        return "#tempo " + self.figure + " " + str(self.speed)

    def milliseconds(self):
        return int(1000000 * 60 * figure_values[self.figure] / (4 * float(self.speed)))

    def children(self):
        return []

class BarDefinition(object):
    def __init__(self, beat, figure, line):
        self.beat = beat
        self.figure = figure

        if beat == 0:
            raise Exception("Cantidad de pulsos incorrecta. Debe ser mayor a 0. Línea {0}".format(line))

        if figure not in figure_values.values():
            raise Exception(
                "Pulso incorrecto. Debe ser el valor de una figura: 1, 2, 4, 8, 16, 32, 64. Línea {0}"
                    .format(line))

    def name(self):
        return "#compas " + self.fraction()

    def fraction(self):
        return str(self.beat) + "/" + str(self.figure)

    def children(self):
        return []

    def get_value(self):
        return self.beat * (1 / float(self.figure))

class Voices(object):
    def __init__(self, voice, other_voices):
        self.voice = voice
        self.other_voices = other_voices

    def name(self):
        return "voces"

    def children(self):
        if self.other_voices == None:
            return [self.voice]
        else:
            return [self.voice, self.other_voices]

    def get_arr_voices(self):
        voices = [self.voice]
        if self.other_voices is not None:
            voices += self.other_voices.get_arr_voices()

        return voices

class Voice(object):
    def __init__(self, voice_number, bars, line):
        self.voice_number = voice_number
        self.bars = bars
        self.line = line

        if voice_number >= 128:
            raise Exception("Instrumento inválido. Debe estar entre 0 y 127. Línea {0}".format(line))

    def name(self):
        return "voz " + str(self.voice_number)

    def children(self):
        return [self.bars]

    def get_bars(self):
        return self.bars.get_arr_bars()

class Bars(object):
    def __init__(self, bar, other_bars):
        self.bar = bar
        self.other_bars = other_bars

    def name(self):
        return "compases"

    def children(self):
        if self.other_bars == None:
            return [self.bar]
        else:
            return [self.bar, self.other_bars]

        return voices

    def get_arr_bars(self):
        bars = self.bar.get_arr_bars()
        if self.other_bars is not None:
            bars += self.other_bars.get_arr_bars()

        return bars

class Bar(object):
    def __init__(self, notes, line):
        self.notes = notes
        self.line = line

    def name(self):
        return "compas"

    def children(self):
        return [self.notes]

    def get_arr_bars(self):
        return [self]

    def get_notes(self):
        return self.notes.get_arr_notes()

    def get_value(self):
        value = 0
        for note in self.get_notes():
            value +=  1 / float(note.get_value())
            if note.dot is not None:
                value += 0.5 / float(note.get_value())

        return value

# Un repetir puede ir en lugar de un compás, por lo tanto también implementa
# get_arr_bars()

class Repeat(object):
    def __init__(self, times, bars, line):
        self.times = times
        self.bars = bars

        if times == 0:
            raise Exception(
                "Número de repeticiones incorrecto. Debe haber al menos 1. Línea {0}".format(line))

    def name(self):
        return "repetir " + str(self.times)

    def children(self):
        return [self.bars]

    def get_arr_bars(self):
        return self.bars.get_arr_bars() * self.times

class Dot(object):
    def __init__(self, dot):
        self.dot = dot

    def name(self):
        return str(self.dot)

    def children(self):
        return []

class NoteModifier(object):
    def __init__(self, note_modifier):
        self.note_modifier = note_modifier

    def name(self):
        return str(self.note_modifier)

    def children(self):
        return []

class Silence(object):
    def __init__(self, figure, dot, other_notes):
        self.figure = figure
        self.dot = dot
        self.other_notes = other_notes

    def name(self):
        return "silencio " + self.figure

    def children(self):
        result = []
        if self.dot is not None:
            result.append(self.dot)
        if self.other_notes is not None:
            result.append(self.other_notes)
        return result

    def get_arr_notes(self):
        notes = [self]
        if self.other_notes is not None:
            notes += self.other_notes.get_arr_notes()

        return notes

    def get_value(self):
        return figure_values[self.figure]

class Note(object):
    def __init__(self, note, note_modifier, octave, figure, dot, other_notes, line):
        self.note = note
        self.note_modifier = note_modifier
        self.octave = octave
        self.figure = figure
        self.dot = dot
        self.other_notes = other_notes

        if octave not in range(1,10):
            raise Exception("Octava incorrecta. Debe estar entre 1 y 9. Línea {0}".format(line))

    def name(self):
        return "nota " + self.note + self.note_modifier_name() + " octava " + str(self.octave)
            + self.figure

    def children(self):
        result = []
        if self.note_modifier is not None:
            result.append(self.note_modifier)
        if self.dot is not None:
            result.append(self.dot)
        if self.other_notes is not None:
            result.append(self.other_notes)
        return result

    def get_arr_notes(self):
        notes = [self]
        if self.other_notes is not None:
            notes += self.other_notes.get_arr_notes()

        return notes

    def note_modifier_name(self):
        note_modifier = ""
        if self.note_modifier is not None:
            note_modifier = self.note_modifier.name()

        return note_modifier

    def __str__(self):
        translation_notes = {
            "do": "c",
            "re": "d",
            "mi": "e",
            "fa": "f",
            "sol": "g",
            "la": "a",
            "si": "b"
        }

        return translation_notes[self.note] + self.note_modifier_name() + str(self.octave)

    def get_value(self):
        return figure_values[self.figure]
\end{verbatim}

\newpage

Y a continuación se muestra parte del código de $musileng$ que contiene la traducción del \textbf{AST} al lenguaje que entiende \textbf{midicomp}:

\begin{verbatim}
def translate_to_txt_midi(ast, output):
    write_header(ast, output)

    for idx, voice in enumerate(ast.get_voices()):
        write_voice(ast, idx, voice, output)


def write_header(ast, output):
    num_tracks = len(ast.get_voices()) + 1
    output.write("MFile 1 {0} {1}\n".format(num_tracks, clicks_per_beat))
    output.write("MTrk\n")
    output.write("000:00:000 TimeSig {0} 24 8\n".format(ast.bar.fraction()))
    output.write("000:00:000 Tempo {0}\n".format(ast.tempo.milliseconds()))
    output.write("000:00:000 Meta TrkEnd\n")
    output.write("TrkEnd\n")

def write_voice(ast, voice_idx, voice, output):
    output.write("MTrk\n")
    output.write("000:00:000 Meta TrkName \"Voz {0}\"\n".format(voice_idx + 1))
    output.write("000:00:000 ProgCh ch={0} prog={1}\n".format(voice_idx + 1, voice.voice_number))

    for bar_idx, bar in enumerate(voice.get_bars()):
        last_bar, last_beat, last_click = write_bar(ast, voice_idx, bar_idx, bar, output)

    output.write("%03d:%02d:%03d Meta TrkEnd\n" % (last_bar, last_beat, last_click))
    output.write("TrkEnd\n")

def write_bar(ast, voice_idx, bar_idx, bar, output):
    beat = 0
    click = 0
    for note in bar.get_notes():
        # note podría ser una nota o un silencio
        if note.__class__.__name__ is 'Note':
            output.write("%03d:%02d:%03d On     ch=%d note=%s vol=70\n" %
                (bar_idx, beat, click, voice_idx + 1, note))

        # La duración se calcula haciendo regla de 3 del valor contra la figura
        # de la definición del compás
        duration = int(clicks_per_beat * ast.bar.figure / float(note.get_value()))
        # Y se agrega media figura si hay un puntillo
        if note.dot is not None:
            duration += int(clicks_per_beat * ast.bar.figure / float(note.get_value() * 2))
        # Se aumentan los clicks, pulsos y el compás, teniendo en cuenta que
        # los clicks son módulo 384 y los pulsos según lo que se indique en la
        # definición del compás
        click += duration
        beat += click / clicks_per_beat
        click %= clicks_per_beat
        bar_idx += beat / ast.bar.beat
        beat %= ast.bar.beat

        if note.__class__.__name__ is 'Note':
            output.write("%03d:%02d:%03d Off    ch=%d note=%s vol=0\n" %
                (bar_idx, beat, click, voice_idx + 1, note))

    return bar_idx, beat, click
\end{verbatim}

\clearpage
