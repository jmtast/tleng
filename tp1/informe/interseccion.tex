Dados dos lenguajes A y B, por las leyes de De Morgan sabemos que vale:

$A \cap B = \overline{\overline{A} \cup \overline{B}}$

Por lo tanto para obtener la intersección de A con B, se hará la unión del complemento de cada uno.

Ya que la unión devuelve un autómata no determinístico, a ese resultado se lo determinizará para luego obtener su complemento.

\subsection{Unión}

Para obtener la unión de 2 autómatas finitos (pueden ser no determinísticos), se utiliza el siguiente algoritmo:

\begin{algorithm}
\begin{algorithmic}[1]
  \Function{union}{$A, B$}

    \State $B \gets \textbf{renombrarEstadosParaEvitarColisiones}(B, estados(A)) $

    \State $\textbf{alfabeto}(A) \gets \textbf{alfabeto}(A) \cup \textbf{alfabeto}(B)$

    \State $\textbf{estados}(A) \gets \textbf{estados}(A) \cup \textbf{estados}(B)$

    \State $\textbf{transiciones}(A) \gets \textbf{transiciones}(A) \cup \textbf{transiciones}(B)$

    \State $\textbf{transiciones}(A) \gets \textbf{transiciones}(A) \cup \{\textbf{q0}(A) \rightarrow_{\lambda} \textbf{q0}(B)\}$

    \State $nuevoQf \gets \textbf{agregarNuevoEstado}(A)$

    \For{$qf$ \textbf{en} \textbf{estadosFinales}(A) $\cup$ \textbf{estadosFinales}(B)}

      \State $\textbf{transiciones}(A) \gets \textbf{transiciones}(A) \cup \{qf \rightarrow_{\lambda} nuevoQf\}$
    \EndFor

    \State $\textbf{estadosFinales}(A) \gets \{nuevoQf\}$

    \State \Return $A$
  \EndFunction
\end{algorithmic}
\end{algorithm}
