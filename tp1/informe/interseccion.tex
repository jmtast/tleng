\section{Ejercicio D - Intersección}

\indent \indent Para obtener el AFD que represente a la intersección de los lenguajes de dos AFD dados A y B, se utilizó que, siguiendo las leyes de De Morgan:\\
\begin{center}
$A \cap B = \overline{\overline{A} \cup \overline{B}}$
\end{center}

\indent Luego, para calcular la intersección, haremos uso de la unión de los complementos de los autómatas. Es menester indicar que, dado que la unión devuelve una AFND, al autómata obtenido se le aplicará el algoritmo de determinación explicado anteriormente para luego calcular su complemento.\\

\subsection{Unión}

\indent \indent Para obtener la unión de dos autómatas finitos, que podrían ser no determinísticos), se utiliza el siguiente algoritmo:\\

\begin{algorithm}
\begin{algorithmic}[1]
  \Function{union}{$A, B$}

    \State $B \gets \textbf{renombrarEstadosParaEvitarColisiones}(B, estados(A)) $

    \State $\textbf{alfabeto}(A) \gets \textbf{alfabeto}(A) \cup \textbf{alfabeto}(B)$

    \State $\textbf{estados}(A) \gets \textbf{estados}(A) \cup \textbf{estados}(B)$

    \State $\textbf{transiciones}(A) \gets \textbf{transiciones}(A) \cup \textbf{transiciones}(B)$

    \State $\textbf{transiciones}(A) \gets \textbf{transiciones}(A) \cup \{\textbf{q0}(A) \rightarrow_{\lambda} \textbf{q0}(B)\}$

    \State $nuevoQf \gets \textbf{agregarNuevoEstado}(A)$

    \For{$qf$ \textbf{en} \textbf{estadosFinales}(A) $\cup$ \textbf{estadosFinales}(B)}

      \State $\textbf{transiciones}(A) \gets \textbf{transiciones}(A) \cup \{qf \rightarrow_{\lambda} nuevoQf\}$
    \EndFor

    \State $\textbf{estadosFinales}(A) \gets \{nuevoQf\}$

    \State \Return $A$
  \EndFunction
\end{algorithmic}
\end{algorithm}
