\section{Ejercicio F - Equivalencia de AFD's}
\indent \indent Dados dos AFD's, se nos requirió determinar si ambos son equivalentes, es decir si reconocen al mismo lenguaje.\\
\indent Para ello, se realizó la intersección entre uno de los autómatas y el complemento del otro. Si para este nuevo autómata que caracteriza la intersección ocurre que hay por lo menos un estado final, se concluye que los dos autómatas no son equivalentes. Esto es claro, dado que si fueran equivalentes estaríamos intersecando un autómata con su complemento, y no debería existir ningún estado final.\\
%no tengo idea de cómo justificar esto. se que A^Ac = vacio, pero no sé como expresarlo en términos de afds. 