\documentclass[10pt, a4paper]{article}

\usepackage[paper=a4paper, left=1.5cm, right=1.5cm, bottom=1.5cm, top=3.5cm]{geometry}
\usepackage[utf8]{inputenc}
\usepackage[spanish]{babel}
\usepackage{caratula}
\usepackage{framed}
\usepackage{ulem}
\usepackage[pdftex]{graphicx}
\usepackage{float}
\usepackage{listings}
\usepackage{hyperref}
\usepackage{amsmath}
\usepackage{amssymb}
\usepackage{algorithm}
\usepackage[noend]{algpseudocode}


%Datos para la caratula
\materia{Teoría de Lenguajes}

\titulo{Trabajo Práctico}

\integrante{Vallejo, Nicolás Agustín}{500/10}{nicopr08@gmail.com}
\integrante{Tastzian, Juan Manuel}{39/10}{jm@tast.com.ar}
\integrante{González, Pablo Gabriel}{146/11}{pablo.gonzalez.alba@gmail.com}
\fecha{29 de Abril de 2015}


\begin{document}
	\maketitle
	\tableofcontents

  \newpage
  \section{Introducción}
  El objetivo de este trabajo pr\'actico es generar una gram\'atica y su respectivo analizador l\'exico y sint\'actico, y procedimientos para mostrar los resultados, para un lenguaje procedural de generaci\'on de contenido por procedimientos para composiciones visuales en tres dimensiones. \\
\\
Se debe poder procesar archivos que est\'en escritos en dicho lenguaje y, de estar escritos correctamente, poder mostrar los resultados, de lo contrario, mostrar el error correspondiente a la falla. \\
\\
Para la realizaci\'on de este trabajo se trabaj\'o en la definici\'on de una gram\'atica para el lenguaje presentado por la c\'atedra, luego, sobre esa gram\'atica se trabaj\'o en la implementaci\'on del analizador l\'exico y el analizador sint\'actico. Por \'ultimo, se trabaj\'o en la implementaci\'on de procesos para mostrar los objetos descriptos por dicha gram\'atica.\\


  \newpage
  \section{Expresión Regular a Autómata Finito Determinístico}
  \section{Ejercicio A - De una expresión regular a un AFD}

\subsection{Generar Autómata}
Para generar el autómata a través de la expresión regular, se va leyendo línea por línea y se separa en casos según lo que se lea:

\begin{itemize}

\item Un caracter: Es el caso base de la recursión. Se genera un autómata [qo] $\rightarrow_{caracter}$ [[q1]]

\item PLUS: Se genera un autómata con las líneas siguientes correspondientes (identificadas por la indentación). Luego, por cada estado final se agrega una transición $\lambda$ al estado inicial.

\item STAR: Lo mismo que para PLUS, pero además se agrega otra transición $\lambda$ del estado inicial al final.

\item OPT: Al autómata generado por las siguientes líneas se le agrega el estado inicial a la lista de estados finales.

\item CONCAT: Se generan todos los autómatas correspondientes (renombrando los estados para que no haya colisiones en ninguno), luego se conectan cada uno con transiciones $\lambda$ desde todos sus estados finales hacia el estado inicial del siguiente.

\item OR: Se generan todos los autómatas correspondientes (renombrando los estados para que no haya colisiones en ninguno), luego del estado inicial del primero se conectan con transiciones $\lambda$ a todos los otros estados iniciales. Se genera un nuevo estado, que se marca como el único final y de todos los otros estados ex-finales se hace una transición $\lambda$ hacia este estado.

\end{itemize}

\subsection{Determinizar}

Para determinizar un AFND-$\lambda$ con Q conjunto de estados y $\sum$ alfabeto, se genera una tabla de Partes(Q) X $\sum$.\\
Se comienza agregando la clausura-$\lambda$ del estado inicial a la tabla y, por cada letra del alfabeto, la clausura-$\lambda$ de los estados a los cuáles se puede llegar, empezando de algún estado en el conjunto de la clausura-$\lambda$ del estado inicial y avanzando por una transición con la letra correpsondiente.\\
Para cada conjunto generado de esta manera, que no se haya calculado previamente, se agrega a la tabla de la misma forma.\\
Cuando ya no quedan conjuntos por procesar, la tabla resultante se transforma a un AFD de la siguiente manera:\\
Los conjuntos generados pasan a ser los estados, el primero será el estado inicial, todos aquellos conjuntos que contengan algún estado final del AFND-$\lambda$ serán los estados finales. Y la tabla indica las transiciones, para cada estado por cada letra a qué otro estado debe ir.


\subsection{Minimizar}

Para minimizar un AFD (en caso de ser AFND-$\lambda$, se utiliza la función para determinizarlo descripta en el punto anterior), se implementó el siguiente algoritmo:

\begin{algorithm}
\begin{algorithmic}[1]
  \Function{minimizar}{$A$}

    \State $A \gets \textbf{removerEstadosNoAlcanzables}(A) $

    \State $\textbf{equivalencia_{k-1}} \gets equivalencia_{0}$

    \State $\textbf{equivalencia_{k}} \gets \textbf{siguienteEquivalencia(equivalencia_{k-1})}$
    
    \While $\textbf{equivalencia_{k-1}} \neq \textbf{equivalencia_{k}}$
    
    		\State $\textbf{equivalencia_{k}} \gets \textbf{siguienteEquivalencia(equivalencia_{k-1})}$

    \EndWhile

    \State \Return $A$
  \EndFunction
\end{algorithmic}
\end{algorithm}

%    minimization pseudocode:
%    eliminate unreachable states (from q0)
%    prepare 0-equivalence -> separate in two sets: final(F) and non-final(NF) states
%    while previous equivalence != current equivalence:
%      while there are unlooked pairs
%        take two states from a previous equivalence partition and compare where they go through each transition
%        if they differ, place them in separate sets
%        else, place them in the same set


  \newpage
  \section{Pertenencia de una cadena al lenguaje}
  \section{Ejercicio A - De una expresión regular a un AFD}

\subsection{Generar Autómata}
Para generar el autómata a través de la expresión regular, se va leyendo línea por línea y se separa en casos según lo que se lea:

\begin{itemize}

\item Un caracter: Es el caso base de la recursión. Se genera un autómata [qo] $\rightarrow_{caracter}$ [[q1]]

\item PLUS: Se genera un autómata con las líneas siguientes correspondientes (identificadas por la indentación). Luego, por cada estado final se agrega una transición $\lambda$ al estado inicial.

\item STAR: Lo mismo que para PLUS, pero además se agrega otra transición $\lambda$ del estado inicial al final.

\item OPT: Al autómata generado por las siguientes líneas se le agrega el estado inicial a la lista de estados finales.

\item CONCAT: Se generan todos los autómatas correspondientes (renombrando los estados para que no haya colisiones en ninguno), luego se conectan cada uno con transiciones $\lambda$ desde todos sus estados finales hacia el estado inicial del siguiente.

\item OR: Se generan todos los autómatas correspondientes (renombrando los estados para que no haya colisiones en ninguno), luego del estado inicial del primero se conectan con transiciones $\lambda$ a todos los otros estados iniciales. Se genera un nuevo estado, que se marca como el único final y de todos los otros estados ex-finales se hace una transición $\lambda$ hacia este estado.

\end{itemize}

\subsection{Determinizar}

Para determinizar un AFND-$\lambda$ con Q conjunto de estados y $\sum$ alfabeto, se genera una tabla de Partes(Q) X $\sum$.\\
Se comienza agregando la clausura-$\lambda$ del estado inicial a la tabla y, por cada letra del alfabeto, la clausura-$\lambda$ de los estados a los cuáles se puede llegar, empezando de algún estado en el conjunto de la clausura-$\lambda$ del estado inicial y avanzando por una transición con la letra correpsondiente.\\
Para cada conjunto generado de esta manera, que no se haya calculado previamente, se agrega a la tabla de la misma forma.\\
Cuando ya no quedan conjuntos por procesar, la tabla resultante se transforma a un AFD de la siguiente manera:\\
Los conjuntos generados pasan a ser los estados, el primero será el estado inicial, todos aquellos conjuntos que contengan algún estado final del AFND-$\lambda$ serán los estados finales. Y la tabla indica las transiciones, para cada estado por cada letra a qué otro estado debe ir.


\subsection{Minimizar}

Para minimizar un AFD (en caso de ser AFND-$\lambda$, se utiliza la función para determinizarlo descripta en el punto anterior), se implementó el siguiente algoritmo:

\begin{algorithm}
\begin{algorithmic}[1]
  \Function{minimizar}{$A$}

    \State $A \gets \textbf{removerEstadosNoAlcanzables}(A) $

    \State $\textbf{equivalencia_{k-1}} \gets equivalencia_{0}$

    \State $\textbf{equivalencia_{k}} \gets \textbf{siguienteEquivalencia(equivalencia_{k-1})}$
    
    \While $\textbf{equivalencia_{k-1}} \neq \textbf{equivalencia_{k}}$
    
    		\State $\textbf{equivalencia_{k}} \gets \textbf{siguienteEquivalencia(equivalencia_{k-1})}$

    \EndWhile

    \State \Return $A$
  \EndFunction
\end{algorithmic}
\end{algorithm}

%    minimization pseudocode:
%    eliminate unreachable states (from q0)
%    prepare 0-equivalence -> separate in two sets: final(F) and non-final(NF) states
%    while previous equivalence != current equivalence:
%      while there are unlooked pairs
%        take two states from a previous equivalence partition and compare where they go through each transition
%        if they differ, place them in separate sets
%        else, place them in the same set


  \newpage
  \section{Intersección}
  \section{Ejercicio D - Intersección}

\indent \indent Para obtener el AFD que represente a la intersección de los lenguajes de dos AFD dados A y B, se utilizó que, siguiendo las leyes de De Morgan:\\
\begin{center}
$A \cap B = \overline{\overline{A} \cup \overline{B}}$
\end{center}

\indent Luego, para calcular la intersección, haremos uso de la unión de los complementos de los autómatas. Es menester indicar que, dado que la unión devuelve una AFND, al autómata obtenido se le aplicará el algoritmo de determinación explicado anteriormente para luego calcular su complemento.\\

\subsection{Unión}

\indent \indent Para obtener la unión de dos autómatas finitos, que podrían ser no determinísticos), se utiliza el siguiente algoritmo:\\

\begin{algorithm}
\begin{algorithmic}[1]
  \Function{union}{$A, B$}

    \State $B \gets \textbf{renombrarEstadosParaEvitarColisiones}(B, estados(A)) $

    \State $\textbf{alfabeto}(A) \gets \textbf{alfabeto}(A) \cup \textbf{alfabeto}(B)$

    \State $\textbf{estados}(A) \gets \textbf{estados}(A) \cup \textbf{estados}(B)$

    \State $\textbf{transiciones}(A) \gets \textbf{transiciones}(A) \cup \textbf{transiciones}(B)$

    \State $\textbf{transiciones}(A) \gets \textbf{transiciones}(A) \cup \{\textbf{q0}(A) \rightarrow_{\lambda} \textbf{q0}(B)\}$

    \State $nuevoQf \gets \textbf{agregarNuevoEstado}(A)$

    \For{$qf$ \textbf{en} \textbf{estadosFinales}(A) $\cup$ \textbf{estadosFinales}(B)}

      \State $\textbf{transiciones}(A) \gets \textbf{transiciones}(A) \cup \{qf \rightarrow_{\lambda} nuevoQf\}$
    \EndFor

    \State $\textbf{estadosFinales}(A) \gets \{nuevoQf\}$

    \State \Return $A$
  \EndFunction
\end{algorithmic}
\end{algorithm}


  \newpage
  \section{Complemento}
  Dado un autómata finito determinístico A, para obtener el complemento del lenguaje que identifica, se utilizará el siguiente algoritmo:

\begin{algorithm}
\begin{algorithmic}[1]
  \Function{complemento}{$A$}
    \State $estadoTrampa \gets \textbf{agregarNuevoEstado}(A)$

    \For{$estado$ \textbf{en} \textbf{estados}(A)}

      \For{$caracter$ \textbf{en} \textbf{alfabeto}(A)}

        \If{$\nexists destino \in \textbf{estados}(A) | (estado \rightarrow_{caracter} destino) \in \textbf{transiciones}(A)$}

          \State $\textbf{transiciones}(A) \gets \textbf{transiciones}(A) \cup \{estado \rightarrow_{caracter} estadoTrampa\}$
        \EndIf
      \EndFor
    \EndFor

    \State $\textbf{estadosFinales}(A) \gets \textbf{estados}(A) \setminus \textbf{estadosFinales}(A)$

    \State \Return $A$
  \EndFunction
\end{algorithmic}
\end{algorithm}


  \newpage
  \section{Equivalencia}
  \section{Ejercicio F - Equivalencia de AFD's}
\indent \indent Dados dos AFD's, se nos requirió determinar si ambos son equivalentes, es decir si reconocen al mismo lenguaje.\\
\indent Dados los autómatas A y A', $L(A) = L(A') \Leftrightarrow L(A) \cap \overline{L(A')} = \emptyset$.
Por lo tanto, se realizó la intersección entre uno de los autómatas y el complemento del otro. Si para este nuevo autómata que caracteriza la intersección ocurre que hay por lo menos un estado final, se concluye que los dos autómatas no son equivalentes. Esto es claro, dado que si fueran equivalentes estaríamos intersecando un autómata con su complemento, y no debería existir ningún estado final.\\



\end{document}
