\section{Ejercicio A - De una expresión regular a un AFD}

\subsection{Generar Autómata}
Para generar el autómata a través de la expresión regular, se va leyendo línea por línea y se separa en casos según lo que se lea:

\begin{itemize}

\item Un caracter: Es el caso base de la recursión. Se genera un autómata [qo] $\rightarrow_{caracter}$ [[q1]]

\item PLUS: Se genera un autómata con las líneas siguientes correspondientes (identificadas por la indentación). Luego, por cada estado final se agrega una transición $\lambda$ al estado inicial.

\item STAR: Lo mismo que para PLUS, pero además se agrega otra transición $\lambda$ del estado inicial al final.

\item OPT: Al autómata generado por las siguientes líneas se le agrega el estado inicial a la lista de estados finales.

\item CONCAT: Se generan todos los autómatas correspondientes (renombrando los estados para que no haya colisiones en ninguno), luego se conectan cada uno con transiciones $\lambda$ desde todos sus estados finales hacia el estado inicial del siguiente.

\item OR: Se generan todos los autómatas correspondientes (renombrando los estados para que no haya colisiones en ninguno), luego del estado inicial del primero se conectan con transiciones $\lambda$ a todos los otros estados iniciales. Se genera un nuevo estado, que se marca como el único final y de todos los otros estados ex-finales se hace una transición $\lambda$ hacia este estado.

\end{itemize}

\subsection{Determinizar}

Para determinizar un AFND-$\lambda$ con Q conjunto de estados y $\sum$ alfabeto, se genera una tabla de Partes(Q) X $\sum$.\\
Se comienza agregando la clausura-$\lambda$ del estado inicial a la tabla y, por cada letra del alfabeto, la clausura-$\lambda$ de los estados a los cuáles se puede llegar, empezando de algún estado en el conjunto de la clausura-$\lambda$ del estado inicial y avanzando por una transición con la letra correpsondiente.\\
Para cada conjunto generado de esta manera, que no se haya calculado previamente, se agrega a la tabla de la misma forma.\\
Cuando ya no quedan conjuntos por procesar, la tabla resultante se transforma a un AFD de la siguiente manera:\\
Los conjuntos generados pasan a ser los estados, el primero será el estado inicial, todos aquellos conjuntos que contengan algún estado final del AFND-$\lambda$ serán los estados finales. Y la tabla indica las transiciones, para cada estado por cada letra a qué otro estado debe ir.


\subsection{Minimizar}

Para minimizar un AFD (en caso de ser AFND-$\lambda$, se utiliza la función para determinizarlo descripta en el punto anterior), se hace lo siguiente:
\begin{enumerate}
	\item Se remueven los estados no alcanzables del autómata A.
	\item Se genera la \textit{0-equivalencia}, que consiste en separar en 2 clases de equivalencia, una con los estados finales y otra con los no finales.
	\item Se calcula la \textit{k-equivalencia} a partir de la \textit{k-1-equivalencia}.
	\item Si la \textit{k-equivalencia} es igual a la \textit{k-1-equivalencia}, se deja de ciclar y se genera a partir de esa clase de equivalencia, el autómata final.
	\item Generación del autómata final:
	\begin{enumerate}
		\item Se crea un estado por cada clase de equivalencia resultante.
		\item Se asigna como estado inicial al estado que representa la clase de equivalencia que contiene al estado inicial original.
		\item Se asignan como finales, todos los estados que representan a las clases de equivalencia que contengan algún estado final original.
		\item En cada clase, se mira el primer estado y se ve a que estados iban en el autómata original. Se agrega una transición al estado que representa la clase en la que dicho estado original correspondía.
	\end{enumerate}
\end{enumerate}
